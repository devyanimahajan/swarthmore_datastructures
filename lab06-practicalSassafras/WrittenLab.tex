% This LaTeX file contains your written lab questions.  You may answer these
% questions just by inserting your answer into this document.
%
% If you're unfamiliar with LaTeX, see the document LearningLaTeX.tex in this
% same directory.  It contains a brief explanation and a few snippets of LaTeX
% code to get you started; in fact, it should have everything you need to
% complete this assignment.
\documentclass{article}

\usepackage{amsmath}
\usepackage{amssymb}
\usepackage{amsthm}
\usepackage{algpseudocode}
\usepackage{algorithmicx}
\usepackage{enumerate}

\newtheorem{claim}{Claim}

\begin{document}
\textbf{Devyani Mahajan}

\textbf{CS 35 Lab 6}
    \section{Inductive Proofs}

    Prove each of the following claims by induction

    \begin{claim}
      The sum of the first $n$ odd numbers is $n^2$.  That is, $\sum\limits_{i=1}^n (2i-1) = n^2$
    \end{claim}

	$P(n): \sum\limits_{i=1}^n (2i-1) = n^2$

    Let us examine the claim for n=1 or P(1):

    $\sum\limits_{i=1}^1 (2i-1) = 1^2$

    $(2*1 -1) = 1^2$

    $1 = 1$

    LHS = RHS

    The claim is true for n=1, P(1) is true


    Assume true for the case of n=k, or P(k)
    $\sum\limits_{i=1}^k (2i-1) = k^2$


    Now let us examine if the proposition holds true for k+1

    Prove P(k+1) is true:

    $\sum\limits_{i=1}^{k+1} (2i-1) = (k+1)^2$

    LHS = $\sum\limits_{i=1}^k (2i-1)+2(k+1)-1$

    = $k^2+2k+2-1$

    = $k^2+2k+1$

    = $(k+1)^2$ = RHS

    LHS = RHS

    P(k+1) is true when P(k) is true


    Therefore P(n) is true for all $n \geq 1$ by principle of mathematical induction



    \begin{claim}
      $\sum\limits_{i=1}^{n} \dfrac{1}{2^i} = 1 - \dfrac{1}{2^n}$
    \end{claim}

    $P(n): \sum\limits_{i=1}^{n} \dfrac{1}{2^i} = 1 - \dfrac{1}{2^n}$

    Let us examine the claim for $n=1 or P(1):$
    
    $\sum\limits_{i=1}^{1} \dfrac{1}{2^i} = 1 - \dfrac{1}{2^1}$
    
    = $\dfrac{1}{2} = 1 - \dfrac{1}{2}$

    The claim is true for $n=1, P(1)$ is true


    Assume true for the case of $n=k, or P(k)$
    
    $\sum\limits_{i=1}^{k} \dfrac{1}{2^i} = 1 - \dfrac{1}{2^k}$

    
    Now let us examine if the proposition holds true for k+1

    Prove P(k+1) is true:
    
    LHS = $\sum\limits_{i=1}^{k+1} \dfrac{1}{2^i}$
    
    = $\sum\limits_{i=1}^{k} \dfrac{1}{2^i} + \dfrac{1}{2^{k+1}}$
    
    = $1 - \dfrac{1}{2^k} + \dfrac{1}{2*2^k}$
    
    = $1 - \dfrac{1}{2^k} + \dfrac{1}{2} \dfrac{1}{2^k}$
    
    = $1 - \dfrac{1}{2^k} (1 - \dfrac{1}{2})$
    
    = $1 - \dfrac{1}{2^k} * \dfrac{1}{2}$
    
    = $1 - \dfrac{1}{2^{k+1}}$ = RHS

    LHS = RHS

    P(k+1) is true when P(k) is true


    Therefore P(n) is true for all $n \geq 1$ by principle of mathematical induction




    \begin{claim}
      $\sum\limits_{i=1}^{n} i^2 = \dfrac{n(n+1)(2n+1)}{6}$
    \end{claim}


    $P(n) : \sum\limits_{i=1}^{n} i^2 = \dfrac{n(n+1)(2n+1)}{6}$
    
    Let us examine the claim for n=1 or P(1):
    
    $\sum\limits_{i=1}^{1} i^2 = \dfrac{1(1+1)(2*1+1)}{6}$
    
    $LHS = 1$
    
    $RHS = 1 = LHS$

    The claim is true for n=1, P(1) is true

    
    Assume true for the case of n=k, or P(k)
    
    $\sum\limits_{i=1}^{k} i^2 = \dfrac{k(k+1)(2k+1)}{6}$

    Now let us examine if the proposition holds true for k+1

    Prove P(k+1) is true:
    
    $ P(k+1): \sum\limits_{i=1}^{k+1} i^2 = \dfrac{(k+1)(k+2)(2(k+1)+1)}{6}
    
    
    $LHS = \sum\limits_{i=1}^{k+1} i^2$
    
    $= {(k+1)}^2 + \sum\limits_{i=1}^{k} i^2 $
    
    $= {(k+1)}^2 + \dfrac{k(k+1)(2k+1)}{6}$
    
    Expanding this, we get:
    
    $\dfrac{2k^3 + 8k^2 + 13k + 6}{6}$
    
    
    $RHS = \dfrac{(k+1)(k+2)(2(k+1)+1)}{6}$
    
    $= \dfrac{(k+1)(k+2)(2k+3)}{6}$
    
    Expanding this, we get:
    
    $\dfrac{2k^3 + 8k^2 + 13k + 6}{6} = LHS$

    LHS = RHS

    P(k+1) is true when P(k) is true


    Therefore P(n) is true for all $n \geq 1$ 
    by principle of mathematical induction




    \vspace{1cm}
    \section{Recursive Invariants}

    The function \texttt{minEven}, given below in pseudocode, takes as input an array $A$ of size $n$ of numbers.  It returns the smallest \textit{even} number in the array.  If no even numbers appear in the array, it returns positive infinity ($+\infty$).  Using induction, prove that the \texttt{minEven} function works correctly.  Clearly state your recursive invariant at the beginning of your proof.

    \begin{verbatim}
Function minEven(A,n)
  If n = 0 Then
    Return +∞
  Else
    Set best To minEven(A,n-1)
    If A[n-1] < best And A[n-1] is even Then
      Set best To A[n-1]
    EndIf
    Return best
  EndIf
EndFunction
    \end{verbatim}

    Let us examine the function \texttt{minEven} for an array A of size 0

    In this case \texttt{minEven(A,0)} will return $+\infty$

    Hence the function \texttt{minEven} works correctly for $n=0$

    This is the recursive invariant ${P(0)}$


    Let us assume that the \texttt{minEven} function works correctly for an array \texttt{A} of size \texttt{k} and returns \texttt{m}

    This is $P(k)$ and is assumed to be true.

    We have to prove that $P(K+1)$ is true given $P(K)$ is true.

    Assume that one more element \texttt{E} is added to the array \texttt{A} increasing its size to \texttt{k+1}

    This element \texttt{E} will be added in the position \texttt{A[n-1]}

    \texttt{E}'s value  can have three cases:

    \textbf{Case 1:} \texttt{E} is odd

    \textbf{Case 2:} \texttt{E} is Even and \texttt{E} \geq \texttt{m}

    \textbf{Case 3:} \texttt{E} is Even and \texttt{E} $<$ \texttt{m}

    Let us now run \texttt{minEven(A,K+1)}:

    $k > 0$, so we go to the \texttt{Else} branch
    \begin{verbatim}
    Set best To minEven(A,n-1): best = m

    If A[n-1] < best And A[n-1] is even Then - Case 3
      Set best To A[n-1]: best = E
    EndIf
    Return best : return m for Case 1 and Case 2, return k for Case 3
    \end{verbatim}
    So \texttt{minEven} works correctly for array of size k+1 if it works for correctly for size k

    for all cases of the added element /texttt{E}

    Hence $P(K+1)$ is true given $P(K)$ is true.

    Hence this function works correctly for all arrays of integers

    using the principle of mathematical induction.



\end{document}
