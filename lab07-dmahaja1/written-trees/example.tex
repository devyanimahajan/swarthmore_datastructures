

 This is a sample BST.


\par\begingroup
\newcommand{\treeDiagramWidth}{0.9\textwidth}
\newcommand{\minimumNodeSize}{8mm}
\newcommand{\verticalSpacing}{10mm}
\begin{center}\begin{tikzpicture}
\tikzstyle{treeNode}=[minimum width=\minimumNodeSize,minimum height=\minimumNodeSize,circle,draw,inner sep=1mm]
\tikzstyle{treeEdge}=[->]
\node[minimum width=\treeDiagramWidth,rectangle] at (\treeDiagramWidth/2,0) {};\node[treeNode] (n-0-0) at ( {(\treeDiagramWidth-(\minimumNodeSize*1))/(1)*(0+0.5) + \minimumNodeSize/2 + \minimumNodeSize*0}, {-0*\verticalSpacing}) {2};
\node[treeNode] (n-1-0) at ( {(\treeDiagramWidth-(\minimumNodeSize*2))/(2)*(0+0.5) + \minimumNodeSize/2 + \minimumNodeSize*0}, {-1*\verticalSpacing}) {1};
\draw[treeEdge] (n-0-0) -- (n-1-0);
\node[treeNode] (n-1-1) at ( {(\treeDiagramWidth-(\minimumNodeSize*2))/(2)*(1+0.5) + \minimumNodeSize/2 + \minimumNodeSize*1}, {-1*\verticalSpacing}) {3};
\draw[treeEdge] (n-0-0) -- (n-1-1);
\end{tikzpicture}\end{center}
\endgroup\par

 I can put slashes into my tree if I want; they don't do anything.


\par\begingroup
\newcommand{\treeDiagramWidth}{0.9\textwidth}
\newcommand{\minimumNodeSize}{8mm}
\newcommand{\verticalSpacing}{10mm}
\begin{center}\begin{tikzpicture}
\tikzstyle{treeNode}=[minimum width=\minimumNodeSize,minimum height=\minimumNodeSize,circle,draw,inner sep=1mm]
\tikzstyle{treeEdge}=[->]
\node[minimum width=\treeDiagramWidth,rectangle] at (\treeDiagramWidth/2,0) {};\node[treeNode] (n-0-0) at ( {(\treeDiagramWidth-(\minimumNodeSize*1))/(1)*(0+0.5) + \minimumNodeSize/2 + \minimumNodeSize*0}, {-0*\verticalSpacing}) {2};
\node[treeNode] (n-1-0) at ( {(\treeDiagramWidth-(\minimumNodeSize*2))/(2)*(0+0.5) + \minimumNodeSize/2 + \minimumNodeSize*0}, {-1*\verticalSpacing}) {1};
\draw[treeEdge] (n-0-0) -- (n-1-0);
\node[treeNode] (n-1-1) at ( {(\treeDiagramWidth-(\minimumNodeSize*2))/(2)*(1+0.5) + \minimumNodeSize/2 + \minimumNodeSize*1}, {-1*\verticalSpacing}) {3};
\draw[treeEdge] (n-0-0) -- (n-1-1);
\end{tikzpicture}\end{center}
\endgroup\par

 Each node decides its parent by the text that is closest to it.  For instance,
 the 3 below is the left child of 4 (and not the right child of 1) because the
 4 is closer to the three.  The blank lines are ignored just like lines with
 symbols are.


\par\begingroup
\newcommand{\treeDiagramWidth}{0.9\textwidth}
\newcommand{\minimumNodeSize}{8mm}
\newcommand{\verticalSpacing}{10mm}
\begin{center}\begin{tikzpicture}
\tikzstyle{treeNode}=[minimum width=\minimumNodeSize,minimum height=\minimumNodeSize,circle,draw,inner sep=1mm]
\tikzstyle{treeEdge}=[->]
\node[minimum width=\treeDiagramWidth,rectangle] at (\treeDiagramWidth/2,0) {};\node[treeNode] (n-0-0) at ( {(\treeDiagramWidth-(\minimumNodeSize*1))/(1)*(0+0.5) + \minimumNodeSize/2 + \minimumNodeSize*0}, {-0*\verticalSpacing}) {2};
\node[treeNode] (n-1-0) at ( {(\treeDiagramWidth-(\minimumNodeSize*2))/(2)*(0+0.5) + \minimumNodeSize/2 + \minimumNodeSize*0}, {-1*\verticalSpacing}) {1};
\draw[treeEdge] (n-0-0) -- (n-1-0);
\node[treeNode] (n-1-1) at ( {(\treeDiagramWidth-(\minimumNodeSize*2))/(2)*(1+0.5) + \minimumNodeSize/2 + \minimumNodeSize*1}, {-1*\verticalSpacing}) {4};
\node[treeNode] (n-2-2) at ( {(\treeDiagramWidth-(\minimumNodeSize*4))/(4)*(2+0.5) + \minimumNodeSize/2 + \minimumNodeSize*2}, {-2*\verticalSpacing}) {3};
\draw[treeEdge] (n-1-1) -- (n-2-2);
\draw[treeEdge] (n-0-0) -- (n-1-1);
\end{tikzpicture}\end{center}
\endgroup\par
